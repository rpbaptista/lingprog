\documentclass[12pt,a4paper]{article}
\usepackage[portuges,brazil]{babel}
\usepackage[T1]{fontenc}
\usepackage[utf8]{inputenc}
\usepackage{amssymb,amsmath,amsfonts}
\usepackage{graphicx}
\usepackage{epstopdf}
\usepackage{fancyhdr}
\usepackage{verbatim}
\usepackage{steinmetz}
\usepackage{epstopdf}
\usepackage{lmodern}
\usepackage{fancybox}
\usepackage{acronym}
\usepackage{commath}
\usepackage{titlesec}

\usepackage{sourcecodepro}


\usepackage[left=3.5cm, right=3.5cm, top=3.00cm, bottom=2.00cm]{geometry}

\usepackage{beramono}% monospaced font with bold variant

\usepackage{listings}
\lstdefinelanguage{VHDL}{
	morekeywords={
		library,use,all,entity,is,port,in,out,end,architecture,of,
		begin,and
	},
	morecomment=[l]--
}

\usepackage{xcolor}
\colorlet{keyword}{blue!100!black!80}
\colorlet{comment}{purple!100!black!90}
\lstdefinestyle{vhdl}{
	language     = VHDL,
	basicstyle   = {\footnotesize \ttfamily},
	keywordstyle = \color{keyword}\bfseries,
	commentstyle = \color{comment}
}





\author{Renata Porciuncula e Vinicius Mesquita}


\titleformat*{\section}{\LARGE\bfseries}

\begin{document}
	
	
	%%%%%%%%%%%%%%%%%%%%%
	\fancypagestyle{plain}{%
		\fancyhf{} % clear all header and footer fields
		\fancyfoot[L]{\includegraphics[width=4.4cm]{figs/minerva-color}}
	
		
		\renewcommand{\headrulewidth}{0pt}
		\renewcommand{\footrulewidth}{0pt}}

	\thispagestyle{plain}
	
	\vspace*{\stretch{1}}
	\noindent
	\begin{center}
		{\Large Universidade Federal do Rio de Janeiro \vspace{\stretch{1}}\\
			\vspace*{\stretch{1}}
			\noindent
			{\Large\textbf{Relatório Técnico}}\\
			\vspace*{\stretch{1}}
			{\LARGE\textbf{Trabalho 1}}\\
			\vspace*{\stretch{1}}
			\hspace{15cm}
			\hfill \parbox{16cm}{\ ~\\ ~\\}
			\hfill \parbox{16cm}{\textbf{Alunos:} Renata P. Baptista e Vinicius M. de Pinho\\}
			\hfill \parbox{16cm}{\textbf{Disciplina:} Linguagens de Programação\\}
			\hfill \parbox{16cm}{\textbf{Professor:} Miguel Elias Mitre Campista\\}
			\hfill \parbox{16cm}{\textbf{Semestre:}  2016.2\\}
			\vspace*{\stretch{1}}
			
		}\end{center}
		
		\vspace*{\stretch{2}}
		
		\begin{center}
			\vspace{\stretch{1}}
			\today\\
		\end{center}
		
%%%%%%%%%%%%%%%%%%%%%%%




%%%%%%%%%%%%%%%%%%%%%%%%%%%%%%%%%%%%%%%%%%%%%%%%%%%%%%%%%%%%%%%%%%%%5
%\section*{Acrônimos}
%
%\begin{acronym}
%
%\end{acronym}
%%%%%%%%%%%%%%%%%%%%%%%%%%%%%%%%%%%%%%%%%%%%%%%%%%%%%%%%%%%%%%%%%%%%5

\cleardoublepage
\section{Introdução}

O desenvolvimento do trabalho para a disciplina de Linguagens de Programação requer um planejamento inicial, referente ao "Trabalho 1". Portanto, o objetivo deste relatório é apresentar a primeira parte do trabalho, onde apresentamos a definição do problema, de como iremos resolvê-lo e do funcionamento interno do programa.


\section{Implementação do Programa}

A ideia do programa a ser desenvolvido é que o mesmo sirva como uma ferramenta que auxilie um professor em suas atividades como docente. O programa gerenciador em C++ servirá de interface entre o  




%\section{Caso de Uso}
%\section{Conclusão}
%\section{Referências}





\end{document}